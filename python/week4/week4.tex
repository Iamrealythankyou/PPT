\documentclass{beamer}
\usepackage{ctex}
\usetheme{Madrid}
\title{python语言程序设计基础}
\author{周恒昇}
\institute{西亚斯学院}
\begin{document}
\begin{frame}
	\frametitle{Outline}
	\begin{itemize}
		\item  容器
		      \begin{itemize}
			      \item 列表
			      \item 元组
			      \item 字典
			      \item 集合
		      \end{itemize}
	\end{itemize}
\end{frame}
\begin{frame}
	\frametitle{元组}
	\framesubtitle{创建元组}
	元组是有序的、不能更改的、可重复的容器。
	\begin{block}{example}
		创建元组thistuple = ("apple", "banana", "cherry", "apple", "cherry")
		使用构造器创建元组thistuple = tuple(("apple", "banana", "cherry", "apple", "cherry"))
	\end{block}
	\begin{alertblock}{定义但个元素的元组}
		创建单元素的元组必须在元素后添加逗号
		thistuple = ("apple",)
		print(type(thistuple))
		thistuple = ("apple")
		print(type(thistuple))
	\end{alertblock}

\end{frame}
\begin{frame}
	\frametitle{元组}
	\framesubtitle{访问元组}
	\begin{itemize}
		\item 索引(正向,反向,截取)
		\item 便利
		\item 筛选
	\end{itemize}
\end{frame}

\begin{frame}
	\frametitle{元组}
	\framesubtitle{添加元素/删除元素}
	因为tuple是不可更改的,如果需要更改tuple中的元素需要将其转化为list类型的变量

\end{frame}

\begin{frame}
	\frametitle{元组}
	\framesubtitle{解包}
	将元组中的元素一次赋给多个变量
	\begin{block}{example}
		fruits = ("apple", "banana", "cherry", "strawberry", "raspberry")

		(green, yellow, *red) = fruits

		print(green)
		print(yellow)
		print(red)
	\end{block}
\end{frame}
\begin{frame}[t]
	\frametitle{元组}
	\framesubtitle{方法}
	\begin{itemize}
		\item count() :输出某个元素在tuple中出的次数
		\item index() :输出某个元素在元组中第一次出现位置的索引值
	\end{itemize}

\end{frame}
\begin{frame}[t]
	\frametitle{集合set}
	\framesubtitle{set创建}
	set是无序、不可更改、不可重复、无索引的容器
	\begin{block}{example}
		thisset = \{"apple", "banana", "cherry"\}
		print(thisset)
		通过构造器创建set thisset = set(("apple", "banana", "cherry")) # note the double round-brackets
	\end{block}
	\begin{alertblock}{set元素不允许重复}
		thisset = \{"apple", "banana", "cherry", True, 1, 2\}\\1和true在set中被认为是值相同的元素
		print(thisset)
	\end{alertblock}
\end{frame}
\begin{frame}[t]
	\frametitle{集合}
	\framesubtitle{访问set}
	\begin{itemize}
		\item 索引
		\item 遍历
		\item 筛选
		\item 检查
		      set没有索引,索引、筛选操作需将set转化为list
		      \begin{block}{检查}
			      thisset = {"apple", "banana", "cherry"}\\
			      print("banana" in thisset)
		      \end{block}

	\end{itemize}
\end{frame}

\begin{frame}[t]
	\frametitle{集合}
	\framesubtitle{向集合中添加元素}
	\begin{itemize}
		\item add()
		\item update()更新原集合、union()=|返回新集合
	\end{itemize}
\end{frame}

\begin{frame}[t]
	\frametitle{集合}
	\framesubtitle{删除集合中的元素}
	\begin{itemize}
		\item remove()
		\item discard()
		\item pop()
		\item clear()
	\end{itemize}
\end{frame}
\begin{frame}[t]
	\frametitle{集合}
	\framesubtitle{集合中的方法}
	\begin{itemize}
		\item intersection() 	\$将两集合中所有重复的元素返回到新集合
		\item intersection	update()不返回新集合,在原集合上更改
		\item difference()	-将在另一个集合中出现过的元素筛掉形成新集合
		\item difference\_update()不创建新集合
		\item symmetric\_difference() \^	\\ 两个集合中所有差异的元素全部同步到新集合
		\item symmetric\_difference\_update()不创建新集合在原集合上更改

	\end{itemize}
	\begin{alertblock}{mind}
		使用运算符号操作集合时,只能操作set类型的值
	\end{alertblock}

\end{frame}
\begin{frame}[t]
	\frametitle{字典}
	\framesubtitle{创建字典}
	字典是有序、可改值、不允许重复的集合
	\begin{block}{example}
		thisdict = {
		"brand": "Ford",
		"model": "Mustang",
		"year": 1964
		}\\
		thisdict = dict(name = "John", age = 36, country = "Norway") :使用构造器创建集合
	\end{block}

\end{frame}
\begin{frame}[t]
	\frametitle{字典}
	\framesubtitle{字典元素的访问}
	\begin{itemize}
		\item get():通过key值访问某个元素的value
		\item keys():返回所有的keys
		\item values():返回所有的values
		\item items():返回所有的items
		\item 遍历字典
		      \begin{itemize}
			      \item for x in dictionary 和keys():遍历keys
			      \item for x in dictionary: dictionary[x] 和values()便利values
			      \item for x,y in dictionary.items: 遍历(key,value)
		      \end{itemize}

	\end{itemize}
	\begin{alertblock}{Attention}
		x =dictionary.keys():在获取字典的keys之后任何对字典的修改都会同步到keys列表,valuse和items也类似
	\end{alertblock}

\end{frame}
\begin{frame}[t]
	\frametitle{字典}
	\framesubtitle{向字典中添加元素}
	\begin{itemize}
		\item dictionary[keys]=values
		\item update(): 函数的值可以是任何iterable类型的变量
	\end{itemize}

\end{frame}
\begin{frame}[t]
	\frametitle{字典}
	\framesubtitle{删除字典中的元素}
	\begin{itemize}
		\item pop(key)删除指定key的元素
		\item popitem()删除最后一个元素
		\item clear()
	\end{itemize}
\end{frame}
\begin{frame}[t]
	\frametitle{字典}
	\framesubtitle{字典的相关方法}
	\begin{itemize}
		\item fromkeys():返回一个所有值相同的字典
	\end{itemize}

\end{frame}
\end{document}
